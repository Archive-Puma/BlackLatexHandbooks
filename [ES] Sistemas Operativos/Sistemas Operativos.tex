% set the document type and the font size
\documentclass[spanish, 12pt]{article}

% use Spanish package
\usepackage[spanish]{babel}

% captions without name
\usepackage{caption}

% set UTF-8 codification
\usepackage[utf8]{inputenc}

% set contents to page using [H]
\usepackage{float}

% insert graphical tree
\usepackage{tikz}
\usepackage{tikz-qtree}

% no equation numeration
\usepackage{amsmath}

% graphs with arrows
\usepackage[all]{xy}

% centering xy graphs
\usepackage{calc}

% cool boxes
\usepackage{fancybox}

\title{Sistemas Operativos}
\author{Diego Enrique Fontán, CosasDePuma}
\date{2017-18}


\begin{document}

	\pagenumbering{gobble}
	\maketitle
	\newpage
	
	\tableofcontents
	\newpage
	
	\pagenumbering{arabic}
	
	\section{Niveles de software}
	
	\vfill
	
		Se denomina \textbf{soporte lógico} o \textbf{software} a todo aquel conjunto de programas asociados a un ordenador.\\
		
		\begin{figure}[H]
			\centering
			\Tree [.Software Aplicaciones Utilidades Control ]
			\caption*{\textit{Niveles de software}}
		\end{figure}
		
		\subsection{Concepto de Sistema Operativo}
		
			Un \textbf{Sistema Operativo} es un programa (o conjunto de subprogramas o módulos) de \underline{control} que tiene como finalidad facilitar el uso del ordenador y conseguir que se utilice eficientemente.\\
			
			\subsubsection{Características del Sistema Operativo}
			
				- Actúa como interfaz entre el usuario y la máquina física.\\
			
				- Controla la ejecución de otros programas.\\
				
				- Gestiona y mantiene ficheros.\\
			
				- Contabiliza la utilización de los recursos.\\
			
				- Protege los datos y los programas.\\
			
				- Gestiona y asigna directamente los recursos hardware.
				
				\begin{equation}
					\notag
					\vdots
				\end{equation}
				
				\newpage
		
			\subsubsection{Funciones principales del Sistema Operativo}
		
				El Sistema Operativo debe inicializar la máquina y preparar el ordenador para su funcionamiento mediante una \textbf{inicialización total} (Initial Program Loading, Bootstrapping) o mediante una \textbf{inicialización parcial}.\\
				
				También servirá de \textbf{máquina extendida o virtual} con el fin de ocultar los detalles del hardware al usuario y proporcionar un entorno más cómodo. De esta forma se logran varios objetivos:\\
				
				- Evitar que la ejecución de los programas se interfieran unos entre otros, proporcionando así \textbf{seguridad}.\\
				
				- Construir recursos virtuales de alto nivel a partir de recursos físicos de de más bajo nivel.\\
				
				Otra de las funciones principales de un Sistema Operativo es la de \textbf{administrar los recursos para su funcionamiento}, ya sea \textit{asignandole} todos los que requiera un programa para su ejecución o \textit{controlando} el uso corrector de estos.\\
				
				Además, es importante que un Sistema Operativo sea \textbf{determinista}, para que un mismo programa ejecutado con los mismo datos de los mismos resultados en cualquier momento y en cualquier ejecución, y que sea \textbf{indeterminista}, para que pueda responder a circunstancias que ocurren de manera impredecible.
				
				\newpage
				
	\section{Tipos de Sistemas}
	
	\vfill
	
		\subsection{Sistemas monolíticos}
		
			No tienen una estructrura definida. Se componen de un conjunto de procedimientos donde cada uno de ellos puede llamar a todos los demás.\\
		
			\centerline{\xymatrix{\\
			& & \bigcirc \ar[d] \ar@/^/[dr] \ar@/_/[dl] &  & \\
			& \bigcirc \ar@/_/[dl] \ar[d] & \bigcirc \ar@/^/[l] \ar@/^/[dr] & \bigcirc \ar[d] \ar@/^/[dr] &   \\
			\bigcirc \ar@/_/[r] & \bigcirc \ar@/_/[l] & \dots & \bigcirc & \bigcirc
			}}
			
			\hfill \break
						
		\subsection{Sistemas en estratos}
		
			Se organiza en una jerarquía de estratos, estando construído cada uno de ellos sobre el otro que tiene menor jerarquía que él.\\
			
			\begin{table}[H]
				\centering
				\begin{tabular}{|c|}
				Operador del SO \\
				Programas de usuario \\
				Administración de E/S \\
				Comunicación operador-procesos \\
				Administración de memoria y tambor \\
				Distribución del procesador y multiprogramación \\
				\end{tabular}
				\caption*{\textit{Ejemplo de sistema THE}}
			\end{table}
			
			\newpage
			
		\subsection{Máquinas virtuales}
		
			Máquinas similares a la máquina real pero de carácter virtual.
			
			El programa de control se ejecuta sobre el propio hardware y ofreec al nivel superior varias máquinas virtuales.
			
			\centerline{\xymatrix{\\
				\fbox{SO 1} \ar@/^/[d] & \fbox{SO 2} \ar@/^/[d] & \fbox{SO 3} \ar@/^/[d] \\
				\fbox{MV 1} \ar@/^/[u] & \fbox{MV 2} \ar@/^/[u] & \fbox{MV 3} \ar@/^/[u] \\
			}}
			\centerline{\xymatrix{
				\fbox{ Monitor de máquinas virtuales } \\
			}}
			\centerline{\xymatrix{
				\doublebox{ Hardware } \\
			}}
			
		\subsection{Modelo Cliente-Servidor}
		
			Su objetivo es minimizar el kernel desplazando el código de todos sus servicios a estratos lo más superiores posibles.\\
			
			Para ello, la mayoría de sus funciones se implementan como procesos del servidor, denominados \textbf{procesos servidores}, de forma que cuando un proceso de usuario, llamado \textbf{proceso cliente}, necesita un servicio del SO, lo que hace es enviar un mensaje al servidor correspondiente (el cual realiza el trabajo y devuelve la respuesta).\\
			
			\begin{figure}[H]
				\centerline{\doublebox{\fbox{Cliente 1} \break \fbox{Servidor de ficheros} \break \fbox{Servidor de la memoria} \break \fbox{Cliente 2} \\[4mm] Kernel}}
			\end{figure}
			
			El kernel. lo único que hace es implementar la comunicación entre cliente-servidor y entre servidor-hardware.\\
			
			\newpage
			
		\subsection{Estructura orientada al objeto}
	
			Se basan en una colección de objetos, donde las funciones del sistema son ficheros, dispositivos, etc.\\
			
			La interacción entre dichos objetos viene determinada por las capacidades que cada uno tenga para actuar con el otro.\\
			
			El kerner es el responsable del mantenimiento de las definiciones de los tipos de objetos soportados y del control de los privilegios de acceso de los mismos.\\
			
			\centerline{\xymatrix{
				\ovalbox{Gestión de memoria} \ar@/_/[dr] & & \ovalbox{Gestión de E/S}\ar@/^/[dl] \\
				& \Ovalbox{ Gestión de Objetos } \ar@/_/[dl] \ar[d] \ar@/^/[dr] & \\
				\textit{\fbox{Procesos}} & \fbox{\textit{Ficheros}} & \fbox{\textit{Dispositivos}} \\
			}}
			
		\subsection{Sistemas híbridos}
		
			Son similares a los sistemas cliente-servidor, aunque añaden ciertas funcionalidades al kernel para que se ejecute más rápido que si permanecen en el espacio de usuario.\\
			
			Se les llama híbridos porque usan mecanismos o conceptos de arquitectura de los sistemas monolíticos y de los sistemas cliente-servidor.\\
			
\end{document}