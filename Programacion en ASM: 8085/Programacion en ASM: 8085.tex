% set the document type and the font size
\documentclass[12pt]{article}

% use Spanish package
\usepackage[spanish]{babel}

% set UTF-8 codification
\usepackage[utf8]{inputenc}

% load TABS package
\usepackage{booktabs}

% set contents to page using [H]
\usepackage{float}

% create the title
\title{\textbf{Programación en ASM: 8085}}
\date{2017-09-24}
\author{Diego Enrique Fontán, CosasDePuma}

% begin the document
\begin{document}
	% print the title
	\maketitle
	% no page number
	\pagenumbering{gobble}
	
	\newpage
	
	% page of contents
	\tableofcontents
	
	\newpage
	
	% first page number
	\pagenumbering{arabic}
	
	% LICENCIA
	
	\textbf{Licencia y propósito de este documento}\\
	
	Este documento, creado por Diego Enrique Fontán (CosasDePuma), pretende ser un apoyo para todo aquel que esté estudiando o esté interesado en aprender a cómo programar en ensamblador 8085. Se da por hecho que el lector ha adquirido una placa similar al procesador 8085 Intel o que se ha apropiado de un software simulador.\\
	
	No calificaré de manual a este escrito, dado que tiene una finalidad más modesta. Es por ello que lo descrito aquí no serán más que un par de apuntes y recordatorios que el autor compartirá con cualquier interesado.\\
	
	La distribución y copia de este documento, así como la edición del mismo, está permitida siempre y cuando el autor original reciba las menciones correspondientes y sin que nadie se lucre a costa de este documento.\\
	
	\newpage
	
	\section{Operaciones de transferencia básicas}
	
		Las operaciones de transferencia son aquellas que nos permiten mover información.\\
		
		Aquí es importante ver que podemos operar de varias maneras:\\
	
		% new table
		% without [H] the table is unlocked
	
		\begin{table}[H]
		
			% center the table
			\centering
		
			%{c center ; l left ; r right ; | separator}
			\begin{tabular}{c|cc}
				\toprule
				Tipo de operación & Destino & Origen     \\
				\midrule
				Inmediato & Registro & Valor             \\
				Por Registro & Registro & Registro       \\
				Directa & Registro & Dirección           \\
				Indirecta & Registro & Puntero (Memoria) \\			
				\bottomrule
			\end{tabular}
		
		\end{table}
	
		\subsection{Inmediato (MVI)}
	
			El método de transferencia inmediato se usa para definir el valor de un Registro asignándole un valor.\\
	
			Este comando ocupa dos bytes: el primero para el opcode de la instrucción y el segundo para el valor en hexadecimal que queremos que sea asignado.\\
	
			\begin{table}[H]
				\centering
				\begin{tabular}{c|c}
					Instrucción & Opcode \\
					\midrule
					MVI A,valor & 3E \\
					MVI B,valor & 06 \\
					MVI C,valor & 0E \\
					MVI D,valor & 16 \\
					MVI E,valor & 1E \\
					MVI H,valor & 26 \\
					MVI L,valor & 2E \\
					MVI M,valor & 36 \\
				\end{tabular}
			\end{table}
	
		\subsection{Por Registro (MOV)}
	
			El método de transferencia por Registro sirve para copiar el valor de un Registro A a un Registro B, siendo el primero el Registro Destino y, el segundo, el Registro Origen.\\
	
			Este comando ocupa un byte y tiene un opcode diferente según los Registros implicados.\\
	
			\begin{table}[H]
				\centering
				\begin{tabular}{c|c||c|c||c|c}
					Instrucción & Opcode & Instrucción & Opcode & Instrucción & Opcode \\
					\midrule
					MOV A,A & 7F & MOV B,A & 47 & MOV C,A & 4F \\
					MOV A,B & 78 & MOV B,B & 40 & MOV C,B & 48 \\
					MOV A,C & 79 & MOV B,C & 41 & MOV C,C & 49 \\
					MOV A,D & 7A & MOV B,D & 42 & MOV C,D & 4A \\
					MOV A,E & 7B & MOV B,E & 43 & MOV C,E & 4B \\
					MOV A,H & 7C & MOV B,H & 44 & MOV C,H & 4C \\
					MOV A,L & 7D & MOV B,L & 45 & MOV C,L & 4D \\
				\end{tabular}
			\end{table}
	
			\begin{table}[H]
				\centering
				\begin{tabular}{c|c||c|c||c|c}
					Instrucción & Opcode & Instrucción & Opcode & Instrucción & Opcode \\
					\midrule
					MOV D,A & 57 & MOV E,A & 5F & MOV H,A & 67 \\
					MOV D,B & 50 & MOV E,B & 58 & MOV H,B & 60 \\
					MOV D,C & 51 & MOV E,C & 59 & MOV H,C & 61 \\
					MOV D,D & 52 & MOV E,D & 5A & MOV H,D & 62 \\
					MOV D,E & 53 & MOV E,E & 5B & MOV H,E & 63 \\
					MOV D,H & 54 & MOV E,H & 5C & MOV H,H & 64 \\
					MOV D,L & 55 & MOV E,L & 5D & MOV H,L & 65 \\
				\end{tabular}
			\end{table}
	
			\begin{table}[H]
				\flushleft
				\hspace{0.3cm}
				\begin{tabular}{c|c}
					Instrucción & Opcode \\
					\midrule
					MOV L,A & 57 \\
					MOV L,B & 50 \\
					MOV L,B & 51 \\
					MOV L,B & 52 \\
					MOV L,B & 53 \\
					MOV L,B & 54 \\
					MOV L,B & 55 \\
				\end{tabular}
			\end{table}
	
		\subsection{Directa}
	
			Las operaciones de transferencia directas que existen en 8085 son algo complicadas, así que se dejará para apartados más avanzados.\\
	
		\subsection{Indirecta (MOV M)}
	
			Las operaciones de transferencia indirectas son aquellas que nos permiten interactuar con la dirección de Memoria a la que apunta el Registro HL.\\
	
			Este comando ocupa un byte y tiene un opcode diferente según los Registros implicados.\\
	
			\begin{table}[H]
				\centering
				\begin{tabular}{c|c||c|c}
					Instrucción & Opcode & Instrucción & Opcode \\
					\midrule
					MOV A,M & 7E & MOV M,A & 77 \\
					MOV B,M & 46 & MOV M,B & 70 \\
					MOV C,M & 4E & MOV M,C & 71 \\
					MOV D,M & 56 & MOV M,D & 72 \\
					MOV E,M & 5D & MOV M,E & 73 \\
					MOV H,M & 66 & MOV M,H & 74 \\
					MOV L,M & 56 & MOV M,L & 75 \\
				\end{tabular}
			\end{table}
	
	\section{Operaciones aritméticas básicas}
	
		Las operaciones aritméticas básicas son aquellas que nos permiten operar el Acumulador.\\
		
		Dentro de esta sección definiremos operaciones tales como Suma/Adición y Resta/Sustracción.\\
		
		De nuevo, estas operaciones pueden ser de varios tipos:\\
		
		\begin{table}[H]
			\centering
			\begin{tabular}{c|cc}
				\toprule
				Tipo de operación & Destino & Origen     \\
				\midrule
				Inmediato & Acumulador & Valor             \\
				Por Registro & Acumulador & Registro       \\
				Directa & Acumulador & Dirección           \\
				Indirecta & Acumulador & Puntero (Memoria) \\			
				\bottomrule
			\end{tabular}
		
		\end{table}
	
		Dado que muchas de estas operaciones implican operaciones con bits de estado (flags), se mostrarán sólamente las más básicas, dejando el resto para apartados más avanzados.\\
		
		\subsection{Por Registro}
		
			Las operaciones aritméticas por Registro nos permiten sumar o restar el valor de un Registro al Acumulador.\\
		
			Estas operaciones ocupan un byte y el opcode varía según el Registro implicado.\\
		
			\subsubsection{Suma/Adición (ADD)}
			
				\begin{table}[H]
					\centering
					\begin{tabular}{c|c}
						Instrucción & Opcode \\
						\midrule
						ADD A & 87 \\
						ADD B & 80 \\
						ADD C & 81 \\
						ADD D & 82 \\
						ADD E & 83 \\
						ADD H & 84 \\
						ADD L & 85 \\
					\end{tabular}
				\end{table}
				
						
			\subsubsection{Resta/Sustracción (SUB)}
			
				\begin{table}[H]
					\centering
					\begin{tabular}{c|c}
						Instrucción & Opcode \\
						\midrule
						SUB A & 97 \\
						SUB B & 90 \\
						SUB C & 91 \\
						SUB D & 92 \\
						SUB E & 93 \\
						SUB H & 94 \\
						SUB L & 95 \\
					\end{tabular}
				\end{table}
		
		\subsection{Inmediato}
		
			Las operaciones aritméticas inmediatas nos permiten añadir o sustraer una cantidad determinada en el Acumulador.\\
			
			Este comando ocupa dos bytes: el primero para el opcode de la instrucción y el segundo para el valor en hexadecimal que queremos que sea añadido.\\
			
			\subsubsection{Suma/Adición (ADI)}
			
				\begin{table}[H]
					\centering
					\begin{tabular}{c|c}
						Instrucción & Opcode \\
						\midrule
						ADI valor & C6 \\
					\end{tabular}
				\end{table}
				
			\subsubsection{Resta/Sustracción (SUI)}
			
				\begin{table}[H]
					\centering
					\begin{tabular}{c|c}
						Instrucción & Opcode \\
						\midrule
						SUI valor & D6 \\
					\end{tabular}
				\end{table}
			
		\subsection{Directa}
	
			Las operaciones aritméticas directas son algo complicadas, así que como ya hemos comentado, se dejará para apartados más avanzados.\\
			
		\subsection{Indirecta}
		
			Las operaciones aritméticas indirectas son aquellas que nos permiten interactuar con la dirección de Memoria a la que apunta el Registro HL.\\
			
			Estas operaciones ocupan un byte.\\
			
			\subsubsection{Suma/Adición (ADD M)}
			
				\begin{table}[H]
					\centering
					\begin{tabular}{c|c}
						Instrucción & Opcode \\
						\midrule
						ADD M & 86 \\
					\end{tabular}
				\end{table}
				
			\subsubsection{Resta/Sustración (SUB M)}
			
				\begin{table}[H]
					\centering
					\begin{tabular}{c|c}
						Instrucción & Opcode \\
						\midrule
						SUB M & 96 \\
					\end{tabular}
				\end{table}
				
	\section{Operaciones aritméticas unitarias básicas}

		Las operaciones aritméticas unitarias son aquellas que modifican en una unidad el valor del Registro con el que se esté trabajando.\\
		
		Aquí podemos distinguir dos operaciones distintas: incremento y decremento.\\
	
		Sólamente existen dos tipos de comandos según su interacción:
	
		\begin{table}[H]
			\centering
			\begin{tabular}{c|cc}
				\toprule
				Tipo de operación & Destino \\
				\midrule
				Inmediato & Registro \\
				Indirecta & Puntero (Memoria) \\			
				\bottomrule
			\end{tabular}
	
		\end{table}
	
		\subsection{Inmediato}
		
			Las operaciones aritméticas unitarias inmediatas son las modificar un Registro directamente.\\
			
			Ocupan un byte y tiene diferentes opcodes según el Registro implicado.\\
	
			\subsubsection{Incremento (INR)}
			
				\begin{table}[H]
					\centering
					\begin{tabular}{c|c}
						Instrucción & Opcode \\
						\midrule
						INR A & 3C \\
						INR B & 04 \\
						INR C & 0C \\
						INR D & 14 \\
						INR E & 1C \\
						INR H & 24 \\
						INR L & 2C \\
					\end{tabular}
				\end{table}
				
			\subsubsection{Decremento (DCR)}
			
				\begin{table}[H]
					\centering
					\begin{tabular}{c|c}
						Instrucción & Opcode \\
						\midrule
						DCR A & 3D \\
						DCR B & 05 \\
						DCR C & 0D \\
						DCR D & 15 \\
						DCR E & 1D \\
						DCR H & 25 \\
						DCR L & 2D \\
					\end{tabular}
				\end{table}
	
\end{document}

