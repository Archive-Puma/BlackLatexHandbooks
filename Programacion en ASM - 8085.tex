% set the document type and the font size
\documentclass[12pt]{article}

% use Spanish package
\usepackage[spanish]{babel}

% set UTF-8 codification
\usepackage[utf8]{inputenc}

% load TABS package
\usepackage{booktabs}

% create the title
\title{\textbf{Programación en ASM: 8085}}
\date{2017-09-24}
\author{Diego Enrique Fontán, CosasDePuma}

% begin the document
\begin{document}
	% print the title
	\maketitle
	% no page number
	\pagenumbering{gobble}
	
	\newpage
	
	% page of contents
	\tableofcontents
	
	\newpage
	
	% first page number
	\pagenumbering{arabic}
	
	% LICENCIA
	
	\textbf{Licencia y propósito de este documento}\\
	
	Este documento, creado por Diego Enrique Fontán (CosasDePuma), pretende ser un apoyo para todo aquel que esté estudiando o esté interesado en aprender a cómo programar en ensamblador 8085. Se da por hecho que el lector ha adquirido una placa similar al procesador 8085 Intel o que se ha apropiado de un software simulador.\\
	
	No calificaré de manual a este escrito, dado que tiene una finalidad más modesta. Es por ello que lo descrito aquí no serán más que un par de apuntes y recordatorios que el autor compartirá con cualquier interesado.\\
	
	La distribución y copia de este documento, así como la edición del mismo, está permitida siempre y cuando el autor original reciba las menciones correspondientes y sin que nadie se lucre a costa de este documento.\\
	
	\newpage
	
	\section{Operaciones de transferencia básicas}
	
	Las operaciones de transferencia son aquellas que nos permiten mover información.\\
	Aquí es importante ver que podemos operar de varias maneras:
	
	% new table
	% without [h!] the table is unlocked
	
	\begin{table}[h!]
		
		% center the table
		\centering
		
		% name the table with a label
		\label{tab:MODOS}
		
		%{c center ; l left ; r right ; | separator}
		\begin{tabular}{c|cc}
			\toprule
			Tipo de operación & Destino & Origen     \\
			\midrule
			Inmediato & Registro & Valor               \\
			Por Registro & Registro & Registro       \\
			Directa & Registro & Dirección         \\
			Indirecta & Registro & Puntero (Memoria) \\			
			\bottomrule
		\end{tabular}
		
	\end{table}
	
	\subsection{Inmediato (MVI)}
	
	El método de transferencia inmediato se usa para definir el valor de un Registro asignándole un valor.\\
	
	Este comando ocupa dos bytes: El primero para el opcode de la instrucción y el segundo para el valor en hexadecimal que queremos que sea asignado.
	
		\begin{table}[h!]
		\centering
		\label{tab:MOV3}
		\begin{tabular}{c|c}
			Instrucción & Opcode \\
			\midrule
			MVI A,valor & 3E \\
			MVI B,valor & 06 \\
			MVI C,valor & 0E \\
			MVI D,valor & 16 \\
			MVI E,valor & 1E \\
			MVI H,valor & 26 \\
			MVI L,valor & 2E \\
			MVI M,valor & 36 \\
		\end{tabular}
	\end{table}
	
	\subsection{Por Registro (MOV)}
	
	El método de transferencia por Registro sirve para copiar el valor de un Registro A a un Registro B, siendo el primero el Registro Destino y, el segundo, el Registro Origen.\\
	
	Este comando ocupa 1 byte y tiene un opcode diferente según los Registros implicados.\\
	
	\begin{table}[h!]
		\centering
		\label{tab:MOV1}
		\begin{tabular}{c|c||c|c||c|c}
			Instrucción & Opcode & Instrucción & Opcode & Instrucción & Opcode \\
			\midrule
			MOV A,A & 7F & MOV B,A & 47 & MOV C,A & 4F \\
			MOV A,B & 78 & MOV B,B & 40 & MOV C,B & 48 \\
			MOV A,C & 79 & MOV B,C & 41 & MOV C,C & 49 \\
			MOV A,D & 7A & MOV B,D & 42 & MOV C,D & 4A \\
			MOV A,E & 7B & MOV B,E & 43 & MOV C,E & 4B \\
			MOV A,H & 7C & MOV B,H & 44 & MOV C,H & 4C \\
			MOV A,L & 7D & MOV B,L & 45 & MOV C,L & 4D \\
		\end{tabular}
	\end{table}
	
	\begin{table}[h!]
		\centering
		\label{tab:MOV2}
		\begin{tabular}{c|c||c|c||c|c}
			Instrucción & Opcode & Instrucción & Opcode & Instrucción & Opcode \\
			\midrule
			MOV D,A & 57 & MOV E,A & 5F & MOV H,A & 67 \\
			MOV D,B & 50 & MOV E,B & 58 & MOV H,B & 60 \\
			MOV D,C & 51 & MOV E,C & 59 & MOV H,C & 61 \\
			MOV D,D & 52 & MOV E,D & 5A & MOV H,D & 62 \\
			MOV D,E & 53 & MOV E,E & 5B & MOV H,E & 63 \\
			MOV D,H & 54 & MOV E,H & 5C & MOV H,H & 64 \\
			MOV D,L & 55 & MOV E,L & 5D & MOV H,L & 65 \\
		\end{tabular}
	\end{table}
	
	\begin{table}[h!]
		\flushleft
		\hspace{0.3cm}
		\label{tab:MOV3}
		\begin{tabular}{c|c}
			Instrucción & Opcode \\
			\midrule
			MOV L,A & 57 \\
			MOV L,B & 50 \\
			MOV L,B & 50 \\
			MOV L,B & 50 \\
			MOV L,B & 50 \\
			MOV L,B & 50 \\
			MOV L,B & 50 \\
		\end{tabular}
	\end{table}
	
	\subsection {Directa}
	
	Las operaciones de transferencia directas que existen en 8085 son algo complicadas, así que se dejará para apartados más avanzados.\\
	
		\subsection {Indirecta (MOV M)}
	
	Las operaciones de transferencia indirectas son aquellas que nos permiten interactuar con la dirección de Memoria a la que apunta el Registro HL.\\
	
	//MEQUEDEAQUI//
	
\end{document}

